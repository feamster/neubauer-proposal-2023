\section{Need for Collaborative Approach}
\label{sec:collaboration}

This project requires expertise in natural language processing, human-computer
interaction, Internet security, networking, and legal issues concerning online
speech and content moderation.  This project will develop new insights that
reveal the extent of content moderation policies and practices across a wide
spectrum of platforms.  First, the proposed research will build the first
public, large-scale, longitudinal, annotated corpus of content moderation
rules across platforms.  Second, this project will shed light on users'
awareness of and experiences with content moderation rules, helping to inform
the public about how both the rules and policies---and their
implementations---differ both across platforms, and even across different
situations and circumstances on the same platform.  Finally, this project will
gather empirical evidence of how content moderation is implemented across
various platforms, as well as whether such implementations are clear and
consistent. In this vein, the research is an interdisciplinary activity
relying on expertise across a range of disciplines, including large-scale
measurement, natural language processing, human computer interaction, law and
policy; the PIs have deep expertise in all of these areas and a long history
of working together at these disciplinary boundaries.

\begin{itemize}
    \itemsep=-1pt
    \item Marshini Chetty is an expert in human-computer interaction and usable privacy and
security. Her past research has focused on understanding user needs for
managing different aspects of Internet use as well as developing user-facing
tools for improving transparency around Internet performance, privacy, and
misleading online content. 
\item Chenhao Tan is an expert in online communities and
natural language processing.  His past research has involved understanding
multi-community engagement, e.g., how new communities emerge from existing
communities, and developing novel methods for causally understanding the
effect of content moderation.  
\item Nick Feamster is an expert in large-scale
Internet measurement, network security, and Internet censorship. His past
research has involved developing and deploying systems for large-scale
Internet performance measurement, including measurement of Internet
censorship; measurements and characterization of the Internet infrastructure
supporting online disinformation campaigns; and software-supported audit
studies of moderation of political advertising on social media platforms. 
\item Genevieve Lakier at the
University of Chicago Law School is a legal expert in online speech, to
couple our qualitative and quantitative research with legal and policy
analysis.
\end{itemize}


