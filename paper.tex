\documentclass{shortpaper}

\usepackage{times}
\usepackage{url,epsfig, arydshln}
\usepackage{graphicx}
\usepackage{subfigure}
\usepackage{cite}
\newcommand{\ea}{{\em et al.}}
\newcommand{\ie}{{\em i.e.}}
\newcommand{\eg}{{\em e.g.}}

\newtheorem{task}{Task}
\newtheorem{problem}{Problem}[section]

\newcommand{\TITLE}{Understanding and Auditing Content Moderation Policies}
%\def\TITLE{Title Here}
\def\heading{\TITLE, \today}

\begin{document}

\formattitle{\TITLE}{Marshini Chetty, Nick Feamster, Chenhao Tan; University of Chicago} 

\section*{Overview}

\begin{tabular}{p{2in}p{4in}}
Title: & \parbox{4in}{\TITLE} \\ \\
Core Research Team: & 
\parbox{4in}{
Professor Marshini Chetty \\
Professor Nick Feamster (Neubauer Professor) \\
Professor Genevieve Lakier \\
Professor Chenhao Tan\\
Department of Computer Science \\
University of Chicago \\
Chicago, Illinois 60637 \\
}
\\
    Funding Amount Requested: & \$300,000
\end{tabular}

\section*{Abstract}
 
As Internet discourse and content becomes increasingly consolidated, a
relatively small number of platforms now have disproportionate decision-making
power over how content is hosted and disseminated on the Internet. The
practice of deciding whether to publish, remove, or flag content that is
posted by third-party users is typically referred to as {\em content
moderation}.  In certain contexts such as copyright, the concept of content
moderation is well-established. On the other hand, over the past several
decades, the scope of content moderation has expanded considerably, to include
a range of speech, including terrorist speech, hate speech, harassment,
disinformation, and more recently, AI generated content.  In many of these
cases, laws and regulations are less well-established. Furthermore, the
technical capabilities to detect forms of content and speech that run afoul of
norms and platform terms of service (in the absence of laws and regulations)
are also less mature. To compound the problem, the amount of content that
users post to platforms has increased by several orders of magnitude.  
To this end, in this project, we aim to study three questions: (1) What is the
landscape of current content moderation policies? (2) How do users perceive,
understand, and react to content moderation? (3) How do different platforms
implement content moderation in practice? By preforming continual, large-scale
measurements of the content moderation practices, we will study a range of
popular online platforms with user-generated content that vary in terms of
type of media, type of content, and scale of platform. 



\pagebreak

\setcounter{page}{1}

\begin{sloppypar}
\section{Research Questions}
\label{sec:questions}

As Internet discourse and content becomes increasingly consolidated, a
relatively small number of platforms now have disproportionate decision-making
power over how content is hosted and disseminated on the Internet. The
practice of deciding whether to publish, remove, or flag content that is
posted by third-party users is typically referred to as {\em content
moderation}.  In certain contexts such as copyright, the concept of content
moderation is well-established. On the other hand, over the past several
decades, the scope of content moderation has expanded considerably, to include
a range of speech, including terrorist speech, hate speech, harassment,
disinformation, and more recently, AI generated content.  In many of these
cases, laws and regulations are less well-established. Furthermore, the
technical capabilities to detect forms of content and speech that run afoul of
norms and platform terms of service (in the absence of laws and regulations)
are also less mature. To compound the problem, the amount of content that
users post to platforms has increased by several orders of magnitude.  Such
scale is incredibly difficult to deal with, particularly for platforms with
fewer employees or operational budgets to invest in complex and expensive
content moderation systems. At the backdrop of this rapidly evolving ecosystem
are platform policies, proposed legislation, regulation, and laws that may not
always square with the technical realities or users’ understanding of these
policies. In light of this evolving---and relatively immature---ecosystem, the
goal of this project is to understand this ecosystem better, from the
perspective of platform policies (i.e., rules). 

To this end, in this project, we aim to study three questions: 
\begin{enumerate}
\item {\bf What is the landscape of current content moderation policies?
        (Section~\ref{sec:landscape}).} We propose to build the first
        public, large-scale, longitudinal, annotated corpus of content moderation
        rules across platforms. 
        We have done preliminary work to build a first version of such corpus and the associated annotations.
        We will expand our preliminary analyses and perform continuous measurements to track
        how these policies change over time. 
        Building on this corpus, we will study whether a platform's stated policies are consistent with both user
        expectation (Section~\ref{sec:user}) and observed practice
        (Section~\ref{sec:auditing}).
    \item {\bf How do users perceive, understand, and react to content
        moderation? (Section~\ref{sec:user}).} One important aspect of a
        trustworthy content moderation system is a user's ability to
        understand a platform's policies and rules. If a platform's policies
        are perceived as inconsistent or arbitrary, the platform may be deemed
        untrustworthy. Furthermore, as previous research has shown~\cite{fiesler_understanding_2015},
        inconsistent behavior concerning moderation and censorship can also
        result in chilling effects, where users self-censor. Thus, it is
        important to understand how users perceive and react to the stated
        policies and rules of different platforms. We will conduct surveys and interviews to gage user understanding of current content
        moderation rules and expectations concerning the enforcement of
        these rules. Content used in the user studies will be drawn
        from the corpus of content that we build and curate in
        Section~\ref{sec:landscape} and~\ref{sec:auditing}.
    \item {\bf How do different platforms implement content moderation in
        practice? (Section~\ref{sec:auditing})} In addition to having policies
        that are consistent with user norms and expectations, a second important aspect of a
        trustworthy content moderation system is predictable, consistent
        application of policies. To this end, we aim to perform continual,
        large-scale measurements of the content moderation practices of
        several online platforms, for a range of different types
        of content. We will perform 
        continual observational measurement studies to observe content moderation behavior
        of different platforms and controlled audit studies where possible.
\end{enumerate}
\noindent
By performing continual, large-scale measurements of the content
moderation practices, we will study a range of popular online
platforms with user-generated content that vary in terms of type of
media, type of content, and scale of platform. This range of platforms
will allow us to understand how moderation policies (and how users perceive,
understand, and react to those policies) differ across platforms. 

\section{Need for Collaborative Approach}
\label{sec:collaboration}

This project requires expertise in natural language processing, human-computer
interaction, Internet security, networking, and legal issues concerning online
speech and content moderation.  This project will develop new insights that
reveal the extent of content moderation policies and practices across a wide
spectrum of platforms.  First, the proposed research will build the first
public, large-scale, longitudinal, annotated corpus of content moderation
rules across platforms.  Second, this project will shed light on users'
awareness of and experiences with content moderation rules, helping to inform
the public about how both the rules and policies---and their
implementations---differ both across platforms, and even across different
situations and circumstances on the same platform.  Finally, this project will
gather empirical evidence of how content moderation is implemented across
various platforms, as well as whether such implementations are clear and
consistent. In this vein, the research is an interdisciplinary activity
relying on expertise across a range of disciplines, including large-scale
measurement, natural language processing, human computer interaction, law and
policy; the PIs have deep expertise in all of these areas and a long history
of working together at these disciplinary boundaries.

\begin{itemize}
    \item Marshini Chetty is an expert in human-computer interaction and usable privacy and
security. Her past research has focused on understanding user needs for
managing different aspects of Internet use as well as developing user-facing
tools for improving transparency around Internet performance, privacy, and
misleading online content. 
\item Chenhao Tan is an expert in online communities and
natural language processing.  His past research has involved understanding
multi-community engagement, e.g., how new communities emerge from existing
communities, and developing novel methods for causally understanding the
effect of content moderation.  
\item Nick Feamster is an expert in large-scale
Internet measurement, network security, and Internet censorship. His past
research has involved developing and deploying systems for large-scale
Internet performance measurement, including measurement of Internet
censorship; measurements and characterization of the Internet infrastructure
supporting online disinformation campaigns; and software-supported audit
studies of moderation of political advertising on social media platforms. 
\item Genevieve Lakier at the
University of Chicago Law School is a legal expert in online speech, to
couple our qualitative and quantitative research with legal and policy
analysis.
\end{itemize}



\section{Impact of Support from Neubauer Collegium}
\label{sec:support}

Humanities involves the processes by which humans think, create, and consume
content---including online content. As humans increasingly shift towards
online interactions in society, the platforms that they use to exchange
thoughts and ideas become paramount. Whereas, prior to the advent of online
social media platforms, such content creation and dissemination was controlled
and moderated in more centralized ways, the advent of online social media
platforms has led to a more decentralized and distributed model of content
creation and dissemination.  The humanistic focus of this proposal involves
how humans interact with each other in these emerging online settings, and in
particular how humans moderate speech in online settings.

The interdisciplinary nature of this proposal---at the nexus of computer
science, law, and policy---makes it difficult for this research to be funded
from conventional sources of funding (e.g., the National Science Foundation).
Typically, the NSF funds research that is squarely in one discipline, such as
computer science, but is often ill-equipped to evaluate proposals that involve
more humanistic and social aspects, such as this one. The PIs have applied to
the NSF for funding on multiple occasions, but have not been successful to
date, largely due to the lack of an appropriate interdisciplinary program at
the NSF for this type of work---particularly the aspects of this work that
involve complex human factors and processes. Commonly, the feedback involves
an inability to grapple with complex human concepts, including the notion of
what it means to be ``manipulated'', as well as the difficulty in evaluating
work that studies the human processes of content moderation.  Nonetheless, in
other cross-disciplinary areas (e.g., Internet equity), the PIs have
successfully obtained funding from the NSF after establishing
interdisicplinary research programs with seed funding from other sources. We
believe that, as in other areas, internal funding from the University of
Chicago can make a significant difference in establishing this research
program as we establish this new area of scholarship.

Support from the Neubauer Collegium for research assistants and postdoctoral
researchers will allow us to recruit and support researchers who can help
execute the research described in this proposal. As the work progresses, we
intend to engage with the Neubauer Collegium, as well as associated faculty in
the Humanities Division, in areas such as the philosophy of technology,
addressing questions concerning how emerging technologies such as online
social media---and the moderation of these platforms---is affecting discourse.
The PIs regularly participate in ongoing interdisciplinary activities on this
topic across campus. For example, Nick Feamster and Genevieve Lakier recently
participated in the University of Chicago Forum for Free Expression event, on
a panel covering the role of AI and social media in free expression online.
Nick Feamster also teaches a course on the topic of Internet censorship and
online speech, and Marshini Chetty teaches a course on misleading online
content.  Both of these courses target and have students enrolled from the
humanities. More support from the Neubauer Collegium will make it possible to
further engage with the humanities community at the University of Chicago,
through courses, interdisciplinary research, and other events.

\section{Work Plan and Method}
\label{sec:work-plan}

\paragraph{Role and Expertise of the Research Team} This collaboration plan is
designed to enable the principal investigators, senior personnel, and students
to effectively work together, communicate, and make decisions that will help
fulfill the project goals. The team includes experts on large-scale network
measurement, human-computer interaction, network security, and natural
language processing. The PIs have extensive experience working
together. Chetty, Feamster, and Tan have been collaborating on research
related to identifying misleading online content since 2020, co-advising
undergraduate, masters, and graduate students in this regard.


\subsection{Part 1: Understanding the Landscape of Platform Moderation}
\label{sec:landscape}

We aim to provide the first systematic overview
of the policies on content moderation across different platforms and compare
them with the related legal requirements on copyright, misinformation, and hate speech if such requirements exist.
This comparison will allow us to understand how content moderation policies
developed by Internet platforms are influenced by legal requirements on
content moderation specified by United States laws. It will also help us to
understand how content moderation policies deviate from each other across
platforms and for the different types of online content we study (copyright,
misinformation, and hate speech).


\begin{table}[t]
    \begin{scriptsize}
    \begin{tabular}{|p{6in}|}
\hline
aliexpress.com, 
amazon.com, 
archive.org,
bbc.co.uk, 
booking.com,
ebay.com, 
etsy.com, 
facebook.com, 
fandom.com, 
flickr.com,
github.com, 
imdb.com, 
imgur.com, 
indeed.com, 
instagram.com,
linkedin.com, 
medium.com, 
msn.com, 
nytimes.com, 
pinterest.com,
pornhub.com, 
quora.com, 
reddit.com, 
researchgate.net, 
sciencedirect.com, 
soundcloud.com,
sourceforge.net, 
spankbang.com, 
stackoverflow.com, 
theguardian.com, 
tiktok.com,
tumblr.com, 
twitch.tv, 
twitter.com, 
vimeo.com, 
w3.org,
washingtonpost.com,
wikipedia.org, 
wordpress.com, 
xhamster.com, 
xnxx.com 
xvideos.com, 
youtube.com
\\ \hline
\end{tabular}
    \end{scriptsize}
\caption{The 43 platforms in our dataset of online content moderation policies (OCMP-43), ordered by their Tranco ranking.}
\label{tab:platformlist}
\end{table}

In preliminary work we have collected content moderation policies from the 43
largest online platforms hosting user-generated content, focusing on policies
around copyright infringement, harmful speech, and misleading content.
\paragraph{Platform selection.} We considered the top $200$ websites from the Tranco
list~\cite{LePochat2019} obtained on June 29, 2022\footnote{The list can be accessed at
\url{https://tranco-list.eu/list/Q9X24/full}}. Of these, we filtered out those
websites that do not host user-generated content, such as \url{google.com},
\url{akamaiedge.net}, and \url{baidu.com}, as well as those which 
are not primarily in English.

\paragraph{Web scraping.} 
Modern web platforms, especially the
most popular sites in our list, lack a consistent structure for distributing
information regarding content moderation policies across a site's
pages. 
Although almost all the platforms do feature a \textit{Terms of Service (ToS)} page
that outline key policies, including those regarding content moderation, many
sites also have separate \textit{Community Guidelines/Standards}, \textit{Help Centers}, and/or
official blog posts. Moreover, a particular platform's policies may be hosted on
another domain entirely---often on parent platforms or customer service
platforms. For instance, many of \url{facebook.com}'s policy pages can be found
on \url{meta.com}, and \url{reddit.com}'s policies are on both
\url{redditinc.com} and \url{reddithelp.com}. We first manually explored the
43 websites of focus and recorded URLs for each site's key policy pages. In most
cases, we found the URLs corresponding to the \textit{ToS}, as well as the \textit{Community
Guidelines/Standards} and \textit{Help Center} if they existed---we refer to these as
\emph{canonical links}.

Informed by the manual exploration, we use a keyword list to identify seed links for each platform by searching these keywords paired with platform names on generic search engines such as \url{google.com}, as well as on the
platforms themselves, if they had a \emph{Help Center} or other searchable database
of links.
We use the set of seed links as the
starting point to explore all web pages on a given platform that may contain topic-related content moderation policies for
that platform. We designed and implemented a custom web scraper, as we found existing open-source
solutions such as Scrapy~\cite{scrapy} and MAXQDA's~\cite{maxqda} built-in
scraper often failed to retrieve text from pages of interest due to bot
blocking, dynamic loading of webpages using JavaScript, as well as direct rate
limiting for queries from a particular Internet Protocol (IP) address.

We ran our scraper from May to July 2023, resulting in text from 8,514
policy pages from all 43 platforms. In some cases, 
the scraper was unable to download
the page source from relevant pages. We manually copied the text from these
115 pages, forming 1.4\% of the total number of pages we scraped. 
We parsed the raw text into sentences and included the 5 sentences before and after the sentence containing the keyword, merging overlapping passages as needed. 
Each passage is then labeled along two axes: which platform the text
came from, and which topic of content moderation the text referred
to---Copyright Infringement, Harmful Speech, or Misleading
Content.
We refer to this dataset as {\em OCMP-43}.

\paragraph{Annotating content moderation policies (work in progress).} 
To provide further structure to the large volume of text from the
content moderation policies, we are working on an
annotation scheme (or ``codebook'') to label relevant sections of text.
Such an approach is commonly used in the social sciences to both categorize
and extract meaning from language-based data~\cite{saldana2021coding}. 
We combine both deductive and inductive approaches to codebook
development. Using the deductive approach, we develop an initial list of
codes that could help answer our hypotheses with regard to the composition of
content moderation policies, and their link to existing legal regimes. We
focus on codes that could capture the \emph{purpose} of a portion of policy
text, especially as it pertained to how a user would interact with it. We are
guided by the principles outlined in Kiesler et al.~\cite{kiesler2012regulating} for healthy online communities, with
emphasis on ``moderation criteria, a chance to argue one's case, and appeal
procedures'' as key components of a comprehensive content moderation policy.
We consider seven preliminary groups of codes: 1) policy justification (e.g., whether extant legal frameworks are references as motivation for policy);
2) moderation criteria (e.g., whether platforms include examples and exceptions);
3) safeguards (e.g., how platforms detect potential candidates to moderate);
4) platform response (e.g., how platforms reviews and enforces content moderation policies);
5) redress/appeal (e.g., venues in which users can appeal to moderation decisions);
6) binding legalese (e.g., how platforms address liability and user rights);
7) signpost (e.g., how platforms link to third-party resources).


\begin{task}[Understanding content moderation policies]
Qualitative and quantitative analyses of content moderation policies.
\end{task}

We have thus far focused on the question of ``Why do platforms say they moderate?'', we will spend the first year further analyzing the following questions with our annotated data: ``How do platforms describe what they moderate?'', ``How do platforms find content that may need moderation?'', ``What happens to flagged content?'', ``What recourse do users have after being moderated?'', and ``How comprehensive are policies across platforms?''
For example, these analyses will help us understand the full approaches platforms take---from engendering policies and finding violative content to responsive enforcement and fielding appeals---and the extent to which their approaches differ across types of platforms and/or types of content.
This step will require close reading of many policies and iteratively improving our coding scheme.
These results will in turn inform the other thrusts of this proposal.

Since annotation can be laborious, we will use large language models for the
initial exercise such as in-context learning with commercial APIs (e.g.,
GPT-4) and fine-tuning open models (e.g., FLAN-T5), and potentially leverage
recent work from Co-PI Tan on few-shot learning from natural language
explanations~\cite{zhou2023flame}.  Note that our dataset is relatively small
and the resultant classifiers can produce noisy labels.  We will thus develop
a machine-in-the-loop approach where algorithms identify candidate
descriptions and candidate labels, and then humans (e.g., our research team)
provide feedback on these results.  Building on Co-PI Tan’s work on the
genealogy between communities~\cite{tan:18}, we will then quantitatively
measure the differences between policies on different platforms for the same
type of content over time.  We hypothesize that there exists wide variability
in how platforms describe their content moderation strategies.  Furthermore,
these differences will depend on the vagueness of the related legal
requirements: the more vague the legal requirements, the greater the variance.
Last but not least, content moderation policies are in constant flux.  We can
also understand the temporal differences of content moderation policies (in
the same platform) building on Task 3.2.



\begin{task}[Building longitudinal datasets]
Building an evolving dataset of content moderation policies.
\end{task}
Our initial effort collected a snapshot of the policies and took three months due to the challenges in scraping.
Building on this work, we plan to conduct a longitudinal study across
platforms by running our scraper at regular intervals and
annotating the text that has changed from the previous run. 
However, the challenges of web scraping will continue to evolve. So, to successfully collect such data over time, we will also need continued maintenance of our scraper to adapt to the new challenges such as increasingly sophisticated bot management techniques.
We plan to open source our scraper to facilitate further research.

An evolving database of content moderation policies could provide
an insightful study of the impact of political and cultural shifts on platform
policies. For instance, there has been a large shift in the content moderation policies of \url{X.com}, formerly known as \url{Twitter} at the time we collected our data. 
Finally, by adjusting the target platforms and topic-wise keyword list, researchers can use the scraper to 
find policy around other topics of interest, such as platform policies on Artificial Intelligence (AI) generated content, or even collect data for training machine learning models.

\paragraph{Expected outcomes and potential risks.}
Given our preliminary work, the main risk lies in how much manual effort will be required in the proposed work.
Because of how painstaking this exercise is, we hope that releasing our code and data will help other researchers study this important topic.
Results in this part will
provide the basis for future observational measurement studies of how content moderation rules are implemented and serve as a dataset of rules from which to draw examples to test in user studies around how users perceive these rules and their effectiveness in Section \ref{sec:user} and \ref{sec:auditing}.


\subsection{Investigating How Users Understand Current Content Moderation Rules }
\label{sec:user}

The second part of this project investigates how to make the content
moderation process easier to understand and more transparent to users. To this
end, this part of the project will study
how users perceive and understand current content moderation rules, as well as
their expectations for how these rules apply. Robinson et al. and Pew Internet
Research Center provide evidence that users do not read Terms of Service (ToS) or
privacy policy documents on online providers and demonstrate that even if you
simplify ToS, this may not by itself increase user understanding of these
documents~\cite{Robinson_Beyond_Agree_2020, PewPrivacyPolicyToS2019}. However,
these researchers also show that repeated exposure to simple ToS rules may
increase understanding of these rules. Similarly, if it is clear what
the content moderation rules are on a platform and how these rules apply,
arguably users will have more trust in a platform's content moderation,
regardless of whether these rules are legally mandated. If, on the other
hand, rules are unclear, hard to find, and ambiguous as to how they apply,
users may feel that moderation decisions are inconsistent and unfair. This
will lead to decreased trust in the platform. Part 2 of the proposal will
allow us to improve the fairness and transparency of content moderation
processes by providing insights into what users know and understand about
current rules whose formats and locations differ across various online
platforms. 

\paragraph{Research questions.}
Our specific research questions are:
\begin{enumerate}
    \item Do users understand the meaning and applicability of the current content moderation rules?
    \item Do users understand how the content moderation rules, as currently worded and enforced, apply to different types of content? 
\end{enumerate}
 
We will use the outcomes of Part~\ref{sec:landscape} to inform the design of our user studies on
content moderation rules by drawing rule excepts from our datasets. 
We hypothesize that
users may understand copyright violations better than hate
speech and misinformation because laws for copyright are more
clear and our preliminary examination of OCMP-43 suggests that these rules are more well articulated as a result. To answer these questions, we will conduct qualitative user studies using interviews and surveys. The results of these studies will inform content moderation
policy design, what needs to be communicated to a user, and how to ensure
users have a recourse for re-engagement or appeals when they are moderated. 

\paragraph{Previous work.} PI Chetty will leverage expertise on conducting qualitative studies, large scale surveys, and experiments with users for this part of the proposal \cite{wyche2013want, romanosky2018understanding, reichel2020have,chetty2007smart, chetty2008getting, kientz2007grow, poole2008more, poole2009computer, kumar2017notellingpasscodes, mathur_endorsements_2018, swart_is_2020, mathur2015mixed,mathur2016they, mathur2018characterizing, chang2017spiders, mathur2019dark}. Additionally, PI Chetty will build on her knowledge of how to develop prototypes for the creation of example content to test in the user studies \cite{chetty2010s, chetty2011my, chetty2012you, mathur2016they, kumar2018codesign, swart2020ad}.


\begin{task}[User understanding]
Evaluating user comprehension of rules.
\end{task}
Our first goal is to show users different platform rules for
misinformation, hate speech, and copyright and ask them about how clear these rules are to them in terms of what they mean and how they apply. This goal will help determine how clear different content
moderation rules are to platform users, how they expect the rules to apply, and whether particular types of rules or formats are more understandable to users. To better understand how users
comprehend existing content regulation rules, we will administer an online
survey with 1,000 online users of various online platforms, drawing on the most popular sites from our OCMP-43 dataset. Sample sizes are estimated based on PI
Chetty's prior work~\cite{mathur_endorsements_2018}. First, to gage users' experiences with different platforms, we will ask about which platforms participants most frequently use. Next, we will ask users questions to gage their awareness of online content moderation practices over different types of content. For instance, we may ask users which types of content they believe there are rules for and why and whether there are any platforms where some rules apply more than others. In addition, we will ask if users have had prior experiences with content moderation posting on various platforms personally or as observers. After the background information has been gathered, we will show users a rule that pertains to copyright, misinformation, or hate speech from a representative sample of OCMP-43 platforms covering a diversity of categories (e.g., news, shopping, social media, art, or streaming sites). We will curate the rule excerpts we show based on the rules
analysis in Section~\ref{sec:landscape} and each user will see rules from each topic area. 

For each rule shown, we will take the most salient excerpt of the rule and present this excerpt to the users. For instance, 
% in our OMCP-43 dataset, 
the eBay member-to-member contact policy on harmful speech has the following excerpt: \textit{``We don't tolerate threats of physical harm using any method including by email or on our public message boards (such as our discussion boards, groups, and other community areas).'}. After showing this eBay excerpt on harmful speech, we will
ask the user to first explain in their own words what they believe the rule
means with respecting to posting content on the platform. For each rule excerpt, we will ask users to rate the rule for clarity, as well as whether they believe they know when this rule applies and why. For instance, users may not know where to draw the line with AI generated content and copyright so this would be a good example to test in the survey. Next, we will ask users to rate how fair they believe the rule is
as well as how actionable they believe the rule is. We will also ask users what recourse users have if their content is removed based on this rule, whether this rule has a legal basis, and why. To ascertain who they believe implements the rules, we will ask users if they
believe the rules are enacted by humans or machines and finally, ask users if
they have any proposed edits to the rule in question and why. Each user will
be shown rules for the different classes of content violations
(misinformation, copyright, and hate speech). We may also conduct a comparative study where we show the same rule from two different platforms and see which is more clear. As a follow-up, we will conduct interviews with a subset of survey respondents (based on prior work, N=30) to gather more detailed data. We will analyze whether some
classes of rules are more understandable than others and whether some rules
are more easily comprehensible by different platform users (e.g., is an eBay
user more likely to understand a hate speech rule than a Reddit user?).


\textit{Analysis:} We will analyze the data using statistical techniques and qualitative data coding for open-ended questions \cite{lazar2017research}. %https://ir.lawnet.fordham.edu/cgi/viewcontent.cgi?article=1618&context=faculty_scholarship). 
Specifically, we will calculate how much users agree about different rules across all users for each topic (misinformation, copyright, and hate speech) and we may calculate if there are statistical differences across users of different platforms for the same kind of online content moderation rules such as misinformation. Interview data will be qualitatively analyzed using thematic coding and multiple coders on transcribed interview data. We will use codes informed by Part 1's rules analysis. The outcomes of this study will include evidence of which classes or rules are easier for users to understand and why and whether comprehension differs depending on the rule topic or presentation and depending on the types of platforms users frequently use. These outcomes will also inform our studies in Part 3.


\if 0
\begin{figure}[t]
        \centering
        \begin{subfigure}{0.4\textwidth}
            \includegraphics[width=\textwidth]{figures/fbsmallmisinfovaccine.jpg}
            \caption{Facebook Misinformation Rule Excerpt.}
            \label{fig:FBmisinforule}
        \end{subfigure}
        \hfill
        \begin{subfigure}{0.4\textwidth}
            \includegraphics[width=\textwidth]{figures/twitch_rule2.jpg}
            \caption{Twitch Copyright Rule Excerpt.}
            \label{fig:Twitchcopyrightrule}
        \end{subfigure}
        
        \caption{Example rules from Facebook and Twitch on misinformation and copyright respectively drawn from OCMP-43.}
        \label{fig:example_rules_OMCP43}
        \end{figure}
\fi



\begin{task}[Evaluating rule implementation]
Determine whether users can apply rules to example content.
\end{task}


\label{sec:rule-implementation}
\begin{table}[t]
    \begin{tabular}{c|p{1.8in}|p{1.8in}|p{1.8in}}
        & {\bf Copyright} & {\bf Hate Speech} & {\bf Misinformation} \\ \hline
        \parbox{0.7in}{Commercial\\ Products} &
        \parbox[c]{\hsize}{\includegraphics[width=1.2in]{figures/obvious/etsycopyrightobvious}} &
        \parbox[c]{\hsize}{\includegraphics[width=1.2in]{figures/kkk.jpg}} &
        \parbox[c]{\hsize}{\includegraphics[width=1.2in]{figures/obvious/etsymisinformationobvious}}
        \\ \hline
    Text & 
        \parbox[c]{\hsize}{\includegraphics[width=0.24\textwidth]{figures/obvious/reddit_copyrightBig.jpg}}  & 
        \parbox[c]{\hsize}{\includegraphics[width=0.24\textwidth]{figures/ebay.jpg}} & 
      \parbox[c]{\hsize}{\includegraphics[width=0.24\textwidth]{figures/facebook_post_misinfo.png}} 
        \\ \hline
    \end{tabular}
    \caption{Examples of content we generated that could violate the content moderation
    rules we observed in OMCP-43, by content type (commercial products, 
    % video,
    text) and by type of violation (copyright, hate speech, misinformation).
    We only show examples for five platforms all of which appear in our preliminary OCMP-43 dataset: Etsy, eBay, Facebook, YouTube, and Reddit.}
    \label{tab:example_content}
\end{table}

To build trust in online platforms, it is important to have rules with clear applicability to posted content. This task will allow us to determine how users expect rules to apply and whether those expectations align with what actually happens on a
platform based on observations in Part 3. We will design an
online survey to evaluate user perceptions of content
moderation rules for copyright, hate speech, and misinformation respectively.
For each set of rules, we will draw at least one rule from the group of rules
pertaining to that topic (e.g., misinformation) and first ask the user to read
the rule, and rate the rule for clarity and fairness. We may also show similar rules drawn from similar types of platforms (e.g., Vimeo versus YouTube versus Twitch versus TikTok).

Second, we will show a user an excerpt of ``content'' and ask them
whether or not this content violates the rule in question and to explain why
or why not the rules applies. Users will also be asked if the content will definitely get moderated, may get moderated, or definitely would not get moderated. Where possible, we will fabricate this content using examples that are quoted in platform rules we have gathered in our preliminary dataset as shown in Table \ref{tab:example_content}. We will vary content to show a spectrum of clearly questionable content to content that is not violating the rule. For instance, a clear copyright violation is show in Table \ref{tab:example_content}, with Mickey Mouse's image on a bag. The table also shows an example of \textit{harmful} speech we drew directly from eBay's member-to-member contact policy and an example of misinformation content drawn from the Facebook help center rules on vaccine misinformation. If we cannot find an appropriate example of content that needs to be moderated directly from OCMP-43 or our extensions in Part 1, we will fabricate this content based on rules in the dataset, taking care to ensure it is not harmful. For instance, based on the copyright rule excerpt from Twitch, we could create a video of a live Twitch stream which uses a Taylor swift song as background music as an example of content violating the rule. We will also vary the types of content shown, for
instance, using a video with misinformation, a Reddit or Facebook post with
harmful speech (as opposed to hate speech), or a product on Etsy with a copyright violation.

Third, we will ask survey takers how the ``content'' was actually regulated (based on real
observations from observational studies in Part 3) and ask them what they think about the
decision. This will allow us to ascertain how users expectations match with
what actually happens on platforms. We will end the survey by asking
users' demographic information and information about what platforms they
typically use and how frequently. Each user will see
rules from all three content moderation rule topics (hate speech, misinformation, copyright) to make a within subjects
comparison of how users respond to different kinds of rules. Rules presented
to the user will be randomized to prevent ordering effects. We will use insights from Part 1 and Part 3 to focus
recruitment and design additional questions as required. 

We will survey sufficient English-speaking users (at least 1,000 users, based
on estimates from prior work~\cite{mathur_endorsements_2018}) in the
United States to compare and contrast how easily users can apply rules
for copyright, misinformation, and hate speech. We will compensate users based on the
survey response time. We will use Prolific to
administer the survey to a survey users, ensuring that we draw
from users of the different target platforms in our OCMP-43 data set. We will also run a more focused interview study to answer the same questions as the survey with online content creators since they are often most affected by content moderation rules. 

\paragraph{Ethical safeguards:} An important \textit{ethical consideration} is
that no content that is detrimental to users' overall well-being will be shown
because we will curate a set of examples to contain only benign examples of
harmful speech, copyright violations, and misinformation on
non-controversial non-hot button topics.  Specifically, we will take care to
avoid using examples or content that is hateful or has direct slurs against a
population group. Even with these precautions, we will inform users clearly of
the types of content they will encounter during the survey at consent time, to
allow them to opt out if they do not wish to participate in the study. We will
also include a trigger warnings on the survey if necessary in the task listing
on Prolific but we will take care not to show examples of content ahead of the
survey. Following the survey (and interviews as applicable), users will also
be debriefed on any potentially misleading content they were exposed to
mitigate post-study effects.


\textit{Analysis:} We will use thematic analysis and ensure that each
transcript is coded by at least two researchers to ensure that we have
consensus with respect to the themes emerging from the data.  We will perform
statistical analysis on the data to determine how well users understand and
respond to different kinds of rules and look at agreement across users for
different classes of rules~\cite{lazar2017research}. We will also use
qualitative data analysis to code any open-ended answers in the
survey~\cite{saldana_coding_2013}. 


\paragraph{Expected outcomes and potential risks.} We will provide
(1)~evidence of how users
understand current content moderation rules for hate speech, misinformation,
and copyright; (2)~evidence of users' expectations of how content moderation
rules apply and evidence of how that matches up with how rules are actually
applied on platforms; and (3)~policy recommendations for amending content
moderation policies to ensure users can comprehend the content regulation
rules for more trustworthy and transparent online endeavors. The challenges of
this part of the project are to generate meaningful content based on the
analysis of the rules and to recruit participants who use different online platforms.
If needed, we will administer the surveys to and conduct interviews with users of a subset of online
platforms to gather a more targeted sample. Moreover, if some rules are more clear
than others for a category of online content, and less likely to expose users to potentially harmful content,
we may focus on one rule type.  

\subsection{Studying Moderation in Practice}
\label{sec:auditing}

To understand the content moderation processes of these platforms, we will
perform both observational studies and active audit studies. Our efforts in
this area builds on our own previous work on performing software-supported
audit studies of moderation of political
advertising~\cite{hounsel2021software}, as well as previous work on auditing
the content moderation policies of platforms, particularly in the area of
algorithmic enforcement of copyright rules~\cite{perel2017black,
gray2020playing}.  

\paragraph{Measurement approach.} We propose to explore the consistency and
transparency of the processes of these platforms using two different
approaches: (1)~Internet-scale measurement; (2)~audit studies.  We describe
these experiments in more detail in the rest of this section.  Our approach in
this project builds on our past work on software-supported audit studies but
takes a more ambitious approach, exploring a broader range of content over a
wider range of platforms augmented by Internet-scale observational studies of the
practices~\cite{hounsel2021software}.  


\paragraph{Research questions.} We plan to study the following questions about
content moderation: 
\begin{enumerate}
    \item {\bf Consistency.} How consistent are the decisions about moderation?  
    \item {\bf Transparency.} Does the platform provide an explanation concerning moderation, to
        the original poster, to other users, or both? Are these explanations
        consistent?  
    \item {\bf Consistency with user expectations.} Is the content moderation
        practice of the platform consistent with user expectations?
\end{enumerate} 

\paragraph{Method and Scope.} 
For each platform, we will monitor new content that is posted 
(either directly by us or by other users of the platform) based a collection
of search terms that include both a fixed set of keywords, as well as
additional keywords based on trending topics and new content. For the case of
copyrighted content, we will rely on a corpus of copyrighted content,
including the titles of artistic works, as well as from other posted
content (e.g., news articles) to identify content that may be copied. 
For misinformation, we will apply named entity extraction to current news stories,
analyze text from the social media sites and accounts of political figures, news agencies,
and government accounts, as well as filter existing corpora of misleading content on various topics~\cite{pathak2019breaking,shahi2021exploratory} to identify keywords for search. To identify posts
involving hate speech, we will rely on keyword search based on extensive existing lists
of keywords \cite{vidgen2020directions}.

\begin{task}[Internet-scale measurement]
    Perform continuous, Internet-scale measurements of content across time in
    the realms of copyright, hate speech, and misinformation.
\end{task}
\noindent
Internet-scale observations of content moderation practices can be performed
continuously, at scale, although they are sometimes more challenging because
moderation events need to be explicitly observed; depending on the nature and
speed of moderation, this type of observation is challenging.  
Yet, past and ongoing work on large-scale Internet measurement of content platforms
suggests that this approach is feasible. In particular, Kumar {\em et al.}
have performed longitudinal analysis of content on Reddit using large-scale
Internet measurement techniques~\cite{kumar2022understanding}. We will take a similar approach, building on
our past work on large-scale Internet web scraping and analysis of content on
a variety of platforms. Specifically, the PIs have recently performed
large-scale content analysis of privacy policies, online content moderation
policies, and opt-out procedures from the California Consumer Privacy
Act~(CCPA), building customized large-scale web measurement tools to
facilitate longitudinal analysis in each case. Although a complete canvassing
of all content platforms is not feasible, we intend to focus our attention on
platforms where the outcomes of content moderation policies may be ambiguous,
based on the results from our user studies.
In this work, we propose to perform continuous, Internet-scale measurements of
content across the platforms we will study in this proposal, to determine
instances of content that is visible, yet flagged, or removed with a
placeholder notice for particular reasons.  
We explore the following questions:

\paragraph{Consistency of decision-making.} Where the effects of content
moderation are visible, we will aim to characterize whether ``similar''
content experiences similar outcomes, as far as moderation is concerned. While
performing this type of experiment is more difficult in an observational
study, we can measure the similarity of certain types of content using
distance metrics either on the content itself or in appropriate embedding spaces, 
and subsequently characterize whether content that has a
similar disposition experiences similar outcomes (e.g., flagging, moderation,
deletion). We can also potentially compare over time.

\paragraph{Presence of explanations.} Although it is difficult in an
observational study to determine whether the original creator or content
poster receives an explanation of subsequent moderation, in the cases where
explanations are visible to the public (e.g., flagged content), we can
observe whether similar content is flagged in similar ways, both within a
platform, across geography, user, and time; as well as across platforms for
similar types of content.

\begin{task}[Audit studies]
    Conduct audit studies to explore how platforms perform
    content moderation for different types of content.
\end{task}
\noindent
To study how platforms implement policies with respect to specific content, as
well as the consistency of policy application across time, user, and
geography, we propose to design and conduct several targeted audit studies
exploring how platforms perform content moderation.

In audit studies, researchers introduce a hypothesis about
decision-making error based on characteristics of the tester and the choice
presented to the decision-making system. A controlled test involves creation
or recruitment of subjects that are similar, to the extent possible, in all
characteristics except for the characteristic under test. Testers are then
randomly assigned to prompt the decision making system to make a choice.
Statistical tests can then reveal differences in decision rates for different
categories of testers and prompts. In the case of this study, the subjects
under test are the platforms themselves (and, specifically, the processes
involved in auditing the content). Since we are testing for consistency of
rule enforcement, and consistency of communication about rules, the audit
studies in this context will involve creation of a particular piece of
content, followed by repeated attempts to post that content under different
circumstances---time, geography, user account, and so forth.

\paragraph{Previous work.} In our previous work, we looked for systematic
errors in Google and Facebook's political advertising policy
systems~\cite{hounsel2021software}.  In this work, we created software to
support an audit of decision-making systems for political advertising policies
pertaining to the United States. The software we developed queried publicly-available APIs
to help generate ads that testers published to targeted
geographically-targeted audiences on each platform. Our software also
simulated a power analysis to determine how many ads needed to be placed for
each tester. To test erroneous removal of non-election product ads, for
example, we chose
products with names that included the surnames of political candidates, based
on reports that Facebook required authorization by the makers of ``Bush's
Beans", an American food company that shares its name with former United
States presidents \cite{rosenberg_facebook_2018}.
Once we designed types of ads to place and determined how many ads should be
placed, we wrote software to help generate ads for Facebook and Google.



\if 0
\begin{figure*}[h!]
  \centering
  \begin{subfigure}[b]{0.2\textwidth}
    \includegraphics[width=\textwidth]{figures/fb_nature_crop.png}
    \caption{Left-leaning parks/parades ad}
    \label{fig:facebook_ad_1}
  \end{subfigure}
  %
  \begin{subfigure}[b]{0.2\textwidth}
    \includegraphics[width=\textwidth]{figures/fb_veterans.png}
    \caption{Right-leaning parks/parades ad}
    \label{fig:facebook_ad_2}
  \end{subfigure}
  %
  \begin{subfigure}[b]{0.2\textwidth}
    \includegraphics[width=\textwidth]{figures/fb_governor_d.png}
    \caption{Left-leaning product ad}
    \label{fig:facebook_ad_3}
  \end{subfigure}
  %
  \begin{subfigure}[b]{0.2\textwidth}
    \includegraphics[width=\textwidth]{figures/fb_governor_r.png}
    \caption{Right-leaning product ad}
    \label{fig:facebook_ad_4}
  \end{subfigure}
    \caption{Examples of Facebook ads that we generated in a previous audit study.}
  \label{fig:facebook_ad_examples}
\end{figure*}

\begin{figure*}[h!]
  \centering
  \begin{subfigure}[b]{0.3\textwidth}
    \includegraphics[width=\textwidth]{figures/google_governor_d.png}
    \caption{Left-leaning product ad}
    \label{fig:google_ad_1}
  \end{subfigure}
  %
  \begin{subfigure}[b]{0.3\textwidth}
    \includegraphics[width=\textwidth]{figures/google_governor_r.png}
    \caption{Right-leaning product ad}
    \label{fig:google_ad_2}
  \end{subfigure}
    \caption{Examples of Google ads that we generated in a previous audit study.}
  \label{fig:google_ad_examples}
\end{figure*}
\fi

This past work not only introduced new techniques for scaling audit studies,
but also found that Facebook, in particular, made many erroneous decisions
concerning the moderation of advertisements when deciding to display these
advertisements to users. For example, we discovered that
Facebook mistakenly prohibited 17.5\% of advertisements for government
websites, and 4.9\% of advertisements for civic events. Overall, the false
positive rate for moderation of advertisements (i.e., incidents where Facebook
mistakenly moderated or deleted a post) was more than 4\% of all ads that we
placed~\cite{hounsel2021software}.  Many of the same aspects of
experimental design, from content creation to statistical methods, apply to
our proposed work.

\paragraph{Proposed work.} We will perform software-supported audit studies
across the proposed platforms to explore the content moderation policies for
the three different types of content we propose to study (i.e., copyright,
misinformation, and hate speech).  These studies will require the automatic
generation of content, as we performed in previous studies. We will generate
and curate content in two ways: (1)~using information gathered from a
large-scale measurement study of moderated content; and (2)~synthetic, manual
generation of content leveraging existing corpora as well as intuitions about
what content should be moderated, based on analysis of policies (Section
\ref{sec:landscape}). 

For this project, we will target the design of our audit studies to complement
the studies in Section~\ref{sec:user}, where we aim to explore user awareness
and comprehension of content moderation rules on various platforms, as well as
specific rule implementation. Specifically,
Section~\ref{sec:rule-implementation} describes a survey-based study, where we
will present users with content and ask them how they think various content
will be moderated on different platforms. {\bf We will use the content that we
generate as part of that study to launch targeted audit studies to determine
how the platforms actually moderate this content---and, as mentioned, whether
the moderation decisions are consistent over time.} Doing so will provide us
with a unique opportunity to gauge not only the consistency of the
decision-making process for different platforms, but also whether the actual
actions and decisions are consistent with user expectations and understanding.

Although it is sometimes suggested that the platforms themselves take part in
this aspect of the research, it should be noted that the participation of a
platform that it is itself under test introduces a potential conflict of
interest and also risks tainting the study (i.e., if a platform knows it is
being audited for certain types of content, it could alter its internal
processes and thus affect the outcomes of the study). Nevertheless, we will
consider approaching several of these platforms to see if they are interested
in an external evaluation of their content moderations. It is possible that
some platforms may be interested in the results of such a study, which may
help improve their processes. If a platform declines participation in such a
process, that itself provides useful information that can guide our
understanding of how platforms view their own content moderation processes.

\paragraph{Consistency of decision-making.} We will perform several
experiments to gauge the consistency of moderation decisions. First, for the
categories of content shown in Table~\ref{tab:example_content}, we will assess
consistency of decision-making {\em over time}, for different timescales. To
test the consistency of human decision-making processes, we will measure
consistency on shorter timescales by posting the same content from different
user accounts on the same day, as well as on multiple days throughout the same
week. To track longer-term evolution (and potential inconsistencies that may
arise as a result of evolving laws and policies), we will also post the same
content at independent times across multiple months. Due to practical
limitations (both the length of this project, as well as evolving programmatic
APIs on platforms), we do not anticipate performing these measurements over
the course of years. In addition to posting the same content on the same
platform over time to gauge the consistency of rule application, we also plan
to post the same content across platforms within the same timeframe, in cases where
the analysis of policies from Section~\ref{sec:landscape} has suggested that
similar content should be subject to the same moderation outcomes.

\paragraph{Presence of explanation.} For each of the above audit experiments,
we also aim to observe whether, if a moderation decision was made, whether an
explanation was offered, to whom the explanation was offered (i.e., only to
the poster of the content, to the general public, or both), and the specificity
of the explanation. 

\paragraph{Consistency with user expectation.} As mentioned, we will design
these software-supported audit studies to use content that is generated as
part of the survey-based studies described in
Section~\ref{sec:rule-implementation}. In doing so, we will be able to
directly compare outcomes of moderation processes---and the explanations users
receive about outcomes, if any---to user expectations. First, we can determine
whether a user's expectation about moderation outcomes is consistent with
their survey responses. Second, if the outcome from the audit study diverges
from a user's expectation of outcomes, we will perform a {\em post hoc} survey
with the user showing them the true outcome and whether they would have a
revised expectation based on the true outcome. Finally, if the content is in
fact moderated {\em and} the user is provided with an explanation of this
moderation decision, we will follow up to determine whether that explanation is consistent
with the user's understanding and expectations.

\paragraph{Expected outcomes and potential risks.} This section will produce
two content-oriented datasets, created for the purposes of the auditing
experiments and the observational studies, respectively. The auditing dataset
will contain a combination of both real and synthetic content akin to
Table~\ref{tab:example_content} that we used in our auditing experiments, to allow
others to repeat the experiments. The observational dataset will contain data
that we measure in practice from the platforms we choose to focus on, based on the
keyword and named entities that we use to focus our searches. We will also
publish the results of our observational and auditing studies and, where
auditing specific rules proves difficult to do (e.g., due to rule ambiguity),
we will also produce policy recommendations indicating the difficulty of
auditing and enforcement. 

\paragraph{Risk mitigation.}
One of the challenges with performing audit studies in the area of content
moderation is that testing whether content should be moderating runs the risk
of possibly introducing objectionable content that users may see, thus
introducing a potential for harm. To mitigate this risk, we will design these
audit studies based on content in each of the areas that we explore that is
closer to ``boundary conditions''. In other words, for content that is notably
harmful or objectionable, then users should expect that content to be removed,
and we also do not stand to learn much from an audit study that tests whether
such content is removed. The more interesting aspects of such audit studies
lie in the gray areas, where content may be objectionable but not necessarily
harmful. Consider the following examples:
\begin{itemize}
    \itemsep=-1pt
    \item For copyrighted content, we might post some of our own
        copyrighted work, where we are the sole copyright owners.
    \item For misinformation, we might post information pertaining to the
        opening of a local business that does not exist (a non-existent
        business cannot be harmed), or similar events that avoid or minimize harm
        to real-world entities.
    \item For hate speech, we might consider posting content that is
        disrespectful or irreverent, but avoids defaming or slandering any
        individual or group.
\end{itemize}
\noindent
Such examples will need refinement, but the general idea is that we will
design audit studies that test the boundaries of what is acceptable, and
whether platforms are consistent in their moderation of such content. The most
interesting cases will be those where outcomes are not immediately obvious
prior to test---and we expect to be able to design other such low-stakes
boundary examples that test the clarity and effectiveness of rules and process
with minimal or no harm to individuals.



\paragraph{Performing audit studies legally and ethically.}
A legal question is whether audit studies constitute a terms of service
violation,
and whether terms of service violations are illegal under the Computer Fraud
and Abuse Act~(CFAA). In March 2020, in {\em Van Buren v. United States}, the
United States Supreme Court ruled that terms of service violations do not
constitute CFAA violations. The ruling had important implications for many
computer science researchers and clarified the legal guidelines for
researchers who study topics such as those we describe in this work. These
rulings affirm the ACM's position (who filed an amicus brief in the case) that
under many circumstances, computer security researchers should not be
prosecuted for violating the terms of service. In {\em Sandvig v. Barr}, the
Supreme Court ruled that terms of service that prevent researchers from
performing certain activities---including activities such as the creation of
fictitious accounts---can chill research (such as that described in this
proposal) and are a violation of First Amendment rights. As such, reading of
past case law, combined with a survey of past work on audit studies, suggest
that the proposed work is within the bounds of current rulings
and case law.
Ethics considerations are naturally more complex, particularly given the
nature of the experiments we propose. According to the Belmont Report,
researchers should abide by several principles: (1) respect for humans; (2)
beneficence; and (3) justice. In the case of this work, obtaining informed
consent (which would fall under respect for humans) is not practical, since
doing so would expose the subjects to the experiment and taint the audit
itself. Nevertheless, we can abide by respect for humans and beneficence by
limiting the risks associated with the nature of the content that we post, as
described above.

\section{Expected Results}
\label{sec:results}


This project will develop new insights that reveal the extent of content
moderation policies and practices across a wide spectrum of platforms. 
First, the proposed research will build the first
public, large-scale, longitudinal, annotated corpus of content moderation
rules across platforms.
Second, this project will shed light on
users' awareness of and experiences with content moderation rules, helping to
inform the public about how both the rules and policies---and their
implementations---differ both across platforms, and even across different
situations and circumstances on the same platform.
Finally, this project will gather empirical evidence of 
how content moderation is implemented across various platforms, as well as
whether such implementations are clear and consistent. In this vein, the
research is an interdisciplinary activity relying on expertise across a range
of disciplines, including large-scale measurement, natural language
processing, human computer interaction, law and policy; the PIs have deep
 expertise in all of these areas and a long history of working together at
 these disciplinary boundaries.


\end{sloppypar}

\pagebreak
\begin{small}
\bibliographystyle{abbrv}
\bibliography{ref}
\end{small}


\end{document}
