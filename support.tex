\section{Impact of Support from Neubauer Collegium}
\label{sec:support}

Humanities involves the processes by which humans think, create, and consume
content---including online content. As humans increasingly shift towards
online interactions in society, the platforms that they use to exchange
thoughts and ideas become paramount. Whereas, prior to the advent of online
social media platforms, such content creation and dissemination was controlled
and moderated in more centralized ways, the advent of online social media
platforms has led to a more decentralized and distributed model of content
creation and dissemination.  The humanistic focus of this proposal involves
how humans interact with each other in these emerging online settings, and in
particular how humans moderate speech in online settings.

The interdisciplinary nature of this proposal---at the nexus of computer
science, law, and policy---makes it difficult for this research to be funded
from conventional sources of funding (e.g., the National Science Foundation).
Typically, the NSF funds research that is squarely in one discipline, such as
computer science, but is often ill-equipped to evaluate proposals that involve
more humanistic and social aspects, such as this one. The PIs have applied to
the NSF for funding on multiple occasions, but have not been successful to
date, largely due to the lack of an appropriate interdisciplinary program at
the NSF for this type of work---particularly the aspects of this work that
involve complex human factors and processes. Commonly, the feedback involves
an inability to grapple with complex human concepts, including the notion of
what it means to be ``manipulated'', as well as the difficulty in evaluating
work that studies the human processes of content moderation.  Nonetheless, in
other cross-disciplinary areas (e.g., Internet equity), the PIs have
successfully obtained funding from the NSF after establishing
interdisicplinary research programs with seed funding from other sources. We
believe that, as in other areas, internal funding from the University of
Chicago can make a significant difference in establishing this research
program as we establish this new area of scholarship.

Support from the Neubauer Collegium for research assistants and postdoctoral
researchers will allow us to recruit and support researchers who can help
execute the research described in this proposal. As the work progresses, we
intend to engage with the Neubauer Collegium, as well as associated faculty in
the Humanities Division, in areas such as the philosophy of technology,
addressing questions concerning how emerging technologies such as online
social media---and the moderation of these platforms---is affecting discourse.
The PIs regularly participate in ongoing interdisciplinary activities on this
topic across campus. For example, Nick Feamster and Genevieve Lakier recently
participated in the University of Chicago Forum for Free Expression event, on
a panel covering the role of AI and social media in free expression online.
Nick Feamster also teaches a course on the topic of Internet censorship and
online speech, and Marshini Chetty teaches a course on misleading online
content.  Both of these courses target and have students enrolled from the
humanities. More support from the Neubauer Collegium will make it possible to
further engage with the humanities community at the University of Chicago,
through courses, interdisciplinary research, and other events.
